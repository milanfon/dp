\documentclass[12pt]{article}
\usepackage[utf8]{inputenc}
\usepackage[czech]{babel}
\usepackage{geometry}
\usepackage{graphicx}
\usepackage{hyperref}
\usepackage{svg}
\usepackage{float}
\usepackage{amssymb}
\usepackage{tikz}
\usepackage{csquotes}
\usepackage[acronym]{glossaries}
\usepackage{multirow}
\usepackage{amsmath}
\usepackage{pdflscape}
\usepackage{multirow}
\usepackage{array}

\geometry{
    a4paper,
    left=30mm,
    right=25mm,
    top=25mm
}

\MakeOuterQuote{"}

\setlength{\parskip}{1em}

% \usepackage[T1]{fontenc}
% \fontfamily{bch}\selectfont
\usepackage{newtxtext,newtxmath}

% Slovník
\makeglossaries
\newacronym{put}{PUT}{Parameterized Unit Test}
\newacronym{gui}{GUI}{Graphical User Interface}
\newacronym{gguf}{GGUF}{GPT-Generated Unified Format}
\newacronym{ggml}{GGML}{GPT-Generated Model Language}
\newacronym{ml}{ML}{Machine Learning - strojové učení}
\newacronym{vcs}{VCS}{Version Control System}
\newacronym{rnn}{RNN}{Recurrent Neural Network - Rekurenční neuronová síť}
\newacronym{cnn}{CNN}{Convolutional Neural Network - Konvoluční neuronová síť}

\newglossaryentry{llm}{
  name={LLM},
  description={Velký jazykový model (LLM - large language model) je typ modelu strojového učení, který je navržen tak, aby rozuměl a generoval text v přirozeném jazyce. Tyto modely jsou trénovány na obrovských datasetech a jsou schopny provádět různé úkoly, jako je textová klasifikace, generování textu, strojový překlad a další}
}

\newglossaryentry{testsmells}{
  name={testové pachy},
  description={Unit Test Smells jsou špatné návyky nebo postupy, které se mohou objevit při psaní jednotkových testů. Tyto "smelly" často vedou k neefektivním nebo nespolehlivým testům a mohou ztížit údržbu testovacího kódu. Příklady zahrnují Duplicated Asserts, Empty Tests a General Fixture},
}

\newglossaryentry{kontrakt}{
  name={kontrakt},
  plural={kontrakty},
  description={V kontextu jednotkových testů odkazuje na formálně definované podmínky nebo pravidla, které musí být dodrženy během vykonávání kódu. Kontrakty mohou specifikovat, co funkce očekává od svých vstupů a jaké výstupy nebo stavy by měla produkce kódu způsobit. Použití kontraktů pomáhá zajistit, že kód se chová podle očekávání a usnadňuje identifikaci chyb, když tyto podmínky nejsou splněny}
}

\newglossaryentry{filtr}{
  name={filtr},
  plural={filtry},
  description={Ve vývoji softwaru a testování se odkazuje na mechanismy nebo kritéria používaná k selektivnímu výběru testovacích vstupů nebo k rozhodování, které výstupy testů jsou relevantní pro další analýzu. Filtry mohou být použity k odstranění redundantních, nelegálních nebo jinak nežádoucích vstupů z procesu generování testů, což zvyšuje efektivitu testování tím, že se zaměřuje pouze na vstupy, které mohou odhalit chyby nebo porušení kontraktů}
}

\newglossaryentry{temperature}{
    name={teplota},
    plural={teploty},
    description={V terminologii strojového učení se jedná o hyperparametr, který ovládá reativitu a náhodnost vstupu modelu. Vyšší teplota vede k méně předvídatelným a kreativnějším výsledkům, kdežto nižší teplota produkuje více konzervativnější výstup.}
}

\newglossaryentry{context}{
    name={kontext},
    description={Rozsah textu nebo dat, který může model zpracovat při generování odpovědí. Funguje jako krátkodobá paměť modelu a určuje, kolik informací může model vzít v úvahu při tvorbě odpovědí.}
}

\newglossaryentry{token}{
    name={token},
    plural={tokeny},
    description={V NLP oblasti se jedná o rozdělení sekvence textu na menší části (například slova, části slov, atd.).}
}

\newglossaryentry{lsp}{
    name={LSP},
    description={Language Server Protocol. Protokol pro jazykový server používaný pro validaci syntaxe, sémantiky nebo formátování zdrojového kódu. Často se pod tímto označením myslí i samotný jazykový sever, který se o tuto analýzu stará.}
}

\newglossaryentry{servlet}{
    name={servlet},
    description={Serverový programový modul napsaný v programovacím jazyce Java, který zpracovává požadavky klientů a implementuje rozhraní Servlet. Přestože servlety mohou reagovat na jakýkoli typ požadavku, nejčastěji se používají k rozšíření aplikací hostovaných webovými servery.}
}

\newglossaryentry{benchmark}{
    name={benchmark},
    description={Standardizovaný test nebo pokus určení k meření a srovnání výkonu sytémů, komponent či aplikací.}
}


%Tikz
\usetikzlibrary{shapes,arrows,quotes,babel,matrix,fit,positioning}
\tikzstyle{block} = [draw, rectangle, inner sep=1em]
\tikzstyle{sum} = [draw, circle]
\tikzstyle{mtitle} = [draw=none, color=gray, inner sep=0pt,font=\bfseries]
\tikzstyle{mmatrix} = [matrix of nodes, nodes=typetag, row sep=1em]
\tikzstyle{typetag} = [draw=gray, inner sep=1ex, anchor=west]

\begin{document}
    % Titulní strana TODO
    \begin{titlepage}
        \begin{center}
            \vspace*{1cm}

            \Large
            \textbf{
                Západočeská univerzita v Plzni\\
                Fakulta aplikovaných věd\\
                Katedra informatiky\\
                }

            \vfill

            \Huge
            \textbf{}\\
            \vspace*{1cm}
            \Large
            {DIPLOMOVÁ PRÁCE}

            \vfill

            \includegraphics[width=0.4\textwidth]{pic/fav.jpg}

            \vspace*{3cm}

            \Large
            \textbf{PLZEŇ, 2024} \hfill \textbf{Milan Horínek}
            
        \end{center}
    \end{titlepage}

    % TODO: Prohlášení
    \hspace{0pt}
    \vfill

    \begin{center}
        \Large\textbf{PROHLÁŠENÍ}
    \end{center}    
    
    \vspace*{0.5cm}
        
    \noindent 
    Předkládám tímto k posouzení a obhajobě bakalářskou práci zpracovanou na
    závěr studia na Fakultě aplikovaných věd Západočeské univerzity v Plzni.
    \\\\
    Prohlašuji, že jsem bakalářskou práci vypracoval samostatně a výhradně
    s použitím odborné literatury a pramenů, jejichž úplný seznam je její součástí.
    \\\\
    V Plzni dne \today \\\\ 
    \begin{tabular}{p{10cm}p{5cm}}
        &   \dotfill \\
        &   \centering Milan Horínek
    \end{tabular}

    \vspace*{3cm}

    \begin{center}
        \Large{\textbf{PODĚKOVÁNÍ}}
    \end{center}


    \vfill
    \pagebreak

    % Abstrakt / Anotace

    \hspace{0pt}
    \vfill

    \begin{center}
        \Large
        \textbf{ANOTACE}
    \end{center}

    \vspace*{0.5cm}

    \noindent Autor se zabývá problematikou přistávání raket a tím, že většina prací na toto téma používá model RL. Autor se však rozhodl použít pro řízení dopředný výpočet přistávací trajektorie. Autor stručně popisuje dvě metody řízení, které byly testovány - zpětnou vazbu a numerické řešení ze simulace. Bylo zjištěno, že druhá metoda je přesnější, ale první je vhodnější pro model 3DoF. Autor navrhuje možná rozšíření práce, například přidání rušivých vlivů, jako je vítr, a zvýšení robustnosti a přesnosti řídicího systému.

    \noindent\textbf{Klíčová slova:} raketa, přistání, VTVL, pinpoint landing

    \vspace*{3cm}

    \begin{center}
        \Large
        \textbf{ABSTRACT}
    \end{center}

    \vspace*{0.5cm}

    \noindent The author discusses the problem of rocket landing, and how most works on the topic use the RL model. However, the author decided to use forward calculation of the landing trajectory for control. The author briefly describes two control methods that were tested – feedback and numerical solution from the simulation. The latter was found to be more accurate, but the former was more suitable for the 3DoF model. The author suggests possible extensions for the work, such as adding disturbances such as wind, and increasing the robustness and accuracy of the control system.

    \noindent\textbf{Key words:} raketa, přistání, VTVL, pinpoint landing

    \vfill
    \pagebreak

    \hspace{0pt}
    \vfill

    \begin{center}
        \Large
        \bf
        ZADÁNÍ
    \end{center}
    \vspace*{5mm}
    \begin{enumerate}
        \item Prozkoumejte dostupné práce na téma řízení polohy rakety a přistávacího manévru
        \item Navrhněte zjednodušený model rakety včetně potřebných příslušných technických prostředků (aktuátorů, senzorů, atd.)
        \item Navrhněte algoritmus řízení letu rakety včetně potřebných letových módů
        \item Implementujte model a regulační smyčku ve vybraném simulačním prostředí
        \item Zhodnoťte získané výsledky
    \end{enumerate}

    \vspace*{3cm}

    \begin{center}
        \Large
        \bf
        ASSIGNMENT
    \end{center}
    \vspace*{5mm}
    \begin{enumerate}
        \item Explore available papers on rocket attitude control and landing maneuver
        \item Create a simplified model of rocket including the necessary technical means (actuators, sensors, etc.)
        \item Design a rocket flight control algorithm including the necessary flight modes
        \item Implement the model and control loop in the chosen simulation environment
        \item Evaluate the obtained results
    \end{enumerate}

    \vfill
    \pagebreak

    % Obsah
    \newpage
    \tableofcontents

    % Slovník
    \newpage
    \printglossary[type=\acronymtype,title=Zkratky]
    \printglossary[title=Slovník pojmů]

    \newpage
    \section{Úvod}
    %TODO Dopsat - co jsou unit testy a jejich předpoklady, co jsou LLM

    \newpage
    \section{Provedená práce v problematice} \label{sec:previouswork}

        \subsection{Předchozí automatizovaná řešení}
        Jazykové modely nebyli prvními pokusy o automatizované generování jednotkových testů. Ještě před nimi existovala spousta metod zahrnující příklady jako \textit{fuzzing, generování náhodných testů řízených zpětnou vazbou, dynamické symbolické exekuce, vyhledávvací a evoluční techniky, parametrické testování}. Zároveň také již na počátku století byli pokusy o vytvoření vlastní neuronové sítě sloužící právě čistě k úkolu testování softwaru. V této sekci je ukázka několika z nich. 

            \subsubsection{Programatická řešení}
            Jedna z používaných programatických automatizovaných metod pro tvrobu jednotkových testů je tzv. \textit{fuzzing}. V rámci těchto testů musí uživatel stále definovat jeho kódovou strukturu, resp. akce, které test bude provádět a jaký výstup očekávat. Automaticky generovaný je pouze vstup tohoto testu. Výhodou zde tedy je, že uživatel nemusí vytvářet maketu vstupních dat testu, která se zde vytvoří automatizovaně. Zůstává zde však problematika, že pro uživatele není kód \emph{black-box}, ale celou jeho strukturu včetně požadovaného výstupu musí sám definovat. 

            Pouze vstupy dokáže také generovat metoda \textit{symbolické exekuce}, která postupně analyzuje chování větvení programu. Začíná bez předchozích znalostí a používá řešič omezení k nalezení vstupů, které prozkoumají nové cesty exekuce. Jakmile jsou testy spuštěny s těmito vstupy, nástroj sleduje cestu, kterou program bere, a aktualizuje svou znalostní bázi (q) s novými podmínkami cesty (p). Tento iterativní proces pokračuje, zpřesňuje sadu známých chování a snaží se maximalizovat pokrytí kódu. Nástroje běžně zvládají různé datové typy a respektují pravidla viditelnosti, používají mock objekty a parametrizované makety k simulaci různých chování a vstupů, čímž zlepšuje proces generování testů, aby odhalily potenciální chyby a zajistily komplexní pokrytí testů. \cite{parizek_symbolic_execution} Tato metoda je implementována například v nástroji IntelliTest v rámci IDE \textit{Visual Studio}. Je používáná v kombinaci s \emph{parametrickými testy}, také označovanými jako \acrshort{put}. Ty na rozdíl od tradičních jednotkových testů, které jsou obvykle uzavřené metody, mohou přijímat libovolnou sadu parametrů. Nástroje se pak snaží automaticky generovat (minimální) sadu vstupů, které plně pokryjí kód dosažitelný z testu. Nástroje jako např. \textit{IntelliTest} automaticky generují vstupy pro \arcshort{put}, které pokrývají mnoho výkonnostních cest testovaného kódu. Každý vstup, který pokrývá jinou výkonnostní cestu, je "serializován" jako jednotkový test. Parametrické testy mohou být také obecné metody, v tom případě musí uživatel specifikovat typy použité k instanci metody. Testy také mohou obsahovat atributy pro očekávané a neočekávané vyjímky. Neočekávané vyjímky vedou k selhání testu. \arcshort{put} tedy do velké míry redukují potřebu uživatelského vstupu pro tvorbu jednotkových testů. \cite{IntelliTestInputGeneration2023} \cite{microsoft2023testgen}

            Pokud zvolíme symbolické řešení vstupu společně s determinovanými vstupy a testovací cestou, vzniká tak hybridní řešení zvané jako \emph{konkolické testovaní} nebo \emph{dynamická symbolická exekuce}. Tento druh testů dokáží tvořit nástroje jako SAGE, KLEE nebo S2E. Problémem tohoto přístupu však je, když program vykazuje nedeterministické chování, kdy tyto metody nebudou schopny určit správnou cestu a zároveň tak ani zaručit dobré pokrytí kódu/větví. Velká míra používání stavových proměnných může vést k vysoké výpočetní náročnosti těchto metod a nenalezení praktického řešení. \cite{engler2006exe} \cite{sen2005cute} \cite{zhou2006safedrive}

            Další metodou je \textit{náhodné generování testů řízené zpětnou vazbou}, která je vylepšením pro generování náhodných testů tím, že zahrnuje zpětnou vazbu získanou z provádění testovacích vstupů v průběhu jejich vytváření. Tato technika postupně buduje vstupy tím, že náhodně vybírá metodu volání a hledá argumenty mezi dříve vytvořenými vstupy. Jakmile je vstup sestaven, je proveden a ověřen proti sadě kontraktů a filtrů. Výsledek provedení určuje, zda je vstup redundantní, nelegální, porušující kontrakt nebo užitečný pro generování dalších vstupů. Technika vytváří sadu testů, které se skládají z jednotkových testů pro testované třídy. Úspěšné testy mohou být použity k zajištění, že kontrakty kódu jsou zachovány napříč změnami programu; selhávající testy (porušující jeden nebo více kontraktů) ukazují na potenciální chyby, které by měly být opraveny. Tato metoda dokáže vytvořit nejen vstup pro test, ale i tělo (kód) testu. Ovšem pro uživatele je stále vhodné znát strukturu kódu. \cite{FeedbackDirectedRT}

            Z programatických metod se zdají být nejpokročilejší \emph{evoluční algoritmy} pro generování sad jednotkových testů, využívající přístup založený na vhodnosti, aby vyvíjely testovací případy, které mají za cíl maximalizovat pokrytí kódu a detekci chyb. Tyto algoritmy mohou autonomně generovat testovací vstupy, které jsou navrženy tak, aby prozkoumávaly různé exekuční cesty v aplikaci. Uživatelé mohou interagovat s vygenerovanými testy jakožto s \textit{black-boxem}. Testy se zaměřují na vstupy a výstupy, aniž by potřebovali rozumět vnitřní logice testovaného systému. Tento aspekt evolučního testování je zvláště výhodný při práci se složitými systémy nebo když zdrojový kód není snadno dostupný. Proces iterativně upravuje testovací případy na základě pozorovaných chování, upravuje vstupy pro efektivnější prozkoumání systému a identifikaci potenciálních defektů. Tato metoda podporuje vysokou úroveň automatizace při generování testů, snižuje potřebu manuálního vstupu a umožňuje komplexní pokrytí testů s menším úsilím. %TODO Přidat citace

            \subsubsection{Neuronové sítě}

    
        \subsection{Vydané publikace}
        Jeden z poměrně nedávno vydaných článků (září 2023) nazvaný "An Empirical Evaluation of Using Large Language Models for Automated Unit Test Generation" \cite{schafer2023empirical} se zabývá využitím velkých jazykových modelů (\gls{llm}) pro automatizované generování jednotkových testů v jazyce JavaScript. Implementovali nástroj s názvem \textbf{TESTPILOT}, který využívá \gls{llm} \textit{gpt3.5-turbo}, \textit{code-cushman-002} od společnosti OpenAI a  také model StarCoder, který vznikl jako komunitní projekt \cite{StarCoder2023}. Vstupní sada pro \gls{llm} obsahovala signatury funkcí, komentáře k dokumentaci a příklady použití. Nástroj byl vyhodnocen na 25 balíčcích npm obsahujících celkem 1684 funkcí API. Vygenerované testy dosáhly pomocí gpt3.5-turbo mediánu pokrytí příkazů 70,2\% a pokrytí větví 52,8\%, čímž překonaly nejmodernější techniku generování testů v jazyce JavaScript zaměřenou na zpětnou vazbu, Nessie.

        Zmíněný model \emph{StarCoder} byl představen v článku "StarCoder: may the source be with you!" \cite{StarCoder2023} z května 2023. Vytvořeny byly konkrétně 2 verze, \textit{StarCoder} a \textit{StarCoderBase}, s 15,5 miliardami parametrů a délkou kontextu 8K. Tyto modely jsou natrénovány na datové sadě nazvané \textit{The Stack}, která obsahuje 1 bilion tokenů z permisivně licencovaných repozitářů GitHub. \textit{StarCoderBase} je vícejazyčný model, který překonává ostatní modely open-source \gls{llm} modely, zatímco \textit{StarCoder} je vyladěná verze speciálně pro Python, která se vyrovná nebo překoná stávající modely zaměřené čistě na Python. Článek poskytuje komplexní hodnocení, které ukazuje, že tyto modely jsou vysoce efektivní v různých úlohách souvisejících s kódem.

        Článek "Exploring the Effectiveness of Large Language Models in Generating Unit Tests" \cite{siddiq2023exploring} z dubna 2023 hodnotí výkonnost tří \gls{llm} - \textit{Codex}, \textit{CodeGen} a \textit{GPT-3.5} - při generování jednotkových testů pro třídy jazyka Java. Studie používá jako vstupní sady dva benchmarky, \href{https://paperswithcode.com/dataset/humaneval-x}{HumanEval} a \href{https://paperswithcode.com/dataset/evosuite-sf110-benchmark}{Evosuite SF110}. Klíčová zjištění ukazují, že \textit{Codex} dosáhl více než 80\% pokrytí v datové sadě \textit{HumanEval}, ale žádný z modelů nedosáhl více než 2\% pokrytí v benchmarku \textit{SF110}. Kromě toho se ve vygenerovaných testech často objevovaly tzv. \gls{testsmells}, jako jsou \textit{duplicitní tvrzení} a \textit{prázdné testy} \cite{testsmells}.

            \subsubsection{Srovnání výsledků}


        \subsection{Modely}
            % TODO Tady asi nějaká předmluva k těm modelům a co za modely se vlastně používá 

            Jedním z často používaných modelů v předchozích pracích je \textit{StarCoder} a \textit{StarCoderBase}, diskutovaný již v sekci \ref{sec:previouswork}. \textit{Base} verze je schopna generovat kód pro více jak 80 programovacích jazyků. Model je navržen pro širokou škálu aplikací obsahující \textit{generování}, \textit{modifikaci}, \textit{doplňování} a \textit{vysvětlování} kódu. Jeho distribuce je volná a licence \textbf{CodeML OpenRAIL-M 0.1} \cite{BigCode2023} umožňuje ho využívat pro množštví aplikací včetně komerčních nebo edukačních. Jeho uživatel však má povinnost uvádět, že výsledný kód byl vygenerován modelem. Licence má své restrikce z obavy tvůrců, protože by model mohl někoho při nepsrávném použití ohrozit. Tyto restrikce se aplikují na na všechny derivace projektů pod touto licencí. Zároveň není kompatibilní s Open-Source licencí právě kvůli těmto restrikcím.
            % TODO Tady nechápu ty restrikce

            Nedávno vydaným modelem je \textit{Code Llama} od společnosti Meta. Jedná se o evoluci jejich jazykového modelu \textit{Llama} specializovaný však čistě na úlohy kódování. Je postaven na platformě \textit{Llama 2} a existuje ve třech variantách: \textit{základní Code Llama}, \textit{Code Llama - Python} a \textit{Code Llama - Instruct}. Model podporuje více programovacích jazyků, včetně jazyků jako Python, C++, Java, PHP, Typescript, C# nebo Bash. Je určen pro úlohy, jako je generování kódu, doplňování kódu a ladění. Code Llama je zdarma pro výzkumné i komerční použití a je uvolněn pod licencí MIT. Uživatelé však musí dodržovat  zásady přijatelného použití, ve kterýcg je uvedeno, že model nelze použít k vytvoření služby, která by konkurovala vlastním službám společnosti Meta. 

            Velmi populárním nástrojem pro generování kódu za pomocí \gls{llm} je \href{https://github.com/features/copilot}{GitHub Copilot}, který je postaven na modelu \textit{codex} od OpenAI. Původní model však byl přestal být zákazníkům nabízen a namísto něj OpenAI doporučuje ke generování kódu využívat chat verze modelů GPT-3.5 a GPT-4. Na architektuře GPT-4 je také postavený nástupce služby Copilot, \href{https://github.com/features/preview/copilot-x}{Copilot X}. Zmíněné modely chat GPT-3.5 a GPT-4 jsou primárně určeny pro generování textu formou chatu. Zvládají však zároveň i dobře generovat kód a jsou vhodné i úlohu generování jednotkových testů. Narozdíl od předchozích modelů však nejsou volně distribuovány a jsou poskytovány pouze jako služba společností OpenAI skrze API nebo je lze hostovat v rámci služby Azure společnosti Microsoft, která zajišťuje větší integritu dat. Jedná se tedy o uzavřený model a jeho uživatelé musí souhlasit s jeho podmínkami použití.

            %TODO Existují i další modely, co by bylo super přidat

            \subsubsection{Srovnání modelů}

            \newpage

            \begin{landscape}
                \centering
                \begin{table}[H]
                    \begin{tabular}{|p{6cm}|c|c|c|c|}
                        \hline
                        \textbf{Práce} & \textbf{Model} & \textbf{Benchmark} & \textbf{Pokrytí testy} & \textbf{Úspěšnost} \\
                        \hline
                        \multirow{3}{6cm}{An Empirical Evaluation of Using Large Language Models for Automated Unit Test Generation} & \textit{gpt-3.5-turbo} & \multirow{3}{*}{Sada NPM balíčků} & 70.2\% & 48\% \\
                         & \textit{code-cushman-002} & & 68.2\% & 47.1\% \\
                         & \textit{StarCoder} & & 54\% & 31.5\% \\
                        \hline
                        \multirow{6}{6cm}{Exploring the Effectiveness of Large Language
Models in Generating Unit Tests} & \multirow{2}{*}{\textit{gpt-3.5-turbo}} & HumanEval & 69.1\% & 52.3\% \\
                         & & SF110 & 0.1\% & 6.9\% \\
                         & \multirow{2}{*}{\textit{CodeGen}} & HumanEval & 58.2\% & 23.9\% \\
                         & & SF110 & 0.5\% & 30.2\% \\
                         & \multirow{2}{*}{\textit{Codex (4k)}} & HumanEval & 87.7\% & 76.7\% \\
                         & & SF110 & 1.8\% & 41.1\% \\
                        \hline
                        Java Unit Testing with AI: An AI-Driven Prototype for Unit Test Generation & \textit{gpt-3.5-turbo} & JUTAI - Zero-shot, temperature: \(0\) & 84.7\% & 71\% \\
                        \hline
                    \end{tabular}
                    \centering
                    \caption{Přehled a srovnání studií}
                    \label{tab:paper_comp}
                \end{table}
                \begin{table}[H]
                    \begin{tabular}{|c|c|c|c|c|c|}
                        \hline
                        \textbf{Model} & \textbf{Otevřenost} & \textbf{Licence} & \textbf{Počet parametrů} & \textbf{Počet programovacích jazyků} & \textbf{Datum vydání} \\
                        \hline
                        \textit{StarCoderBase} & Volně dostupný & CodeML OpenRAIL-M 0.1 & 15.5 miliardy & 80+ & 5/2023 \\
                        \hline
                        \textit{Code Llama} & Volně dostupný & Llama 2 Licence & 34 miliard & 8+ & 8/2023 \\
                        \hline
                        \textit{gpt-3.5-turbo} & Uzavřený & Proprietární & 175 miliard? & ? & 5/2023 \\
                        \hline
                        \textit{gpt-4} & Uzavřený & Proprietární & 1.76 bilionu & ? & 3/2023 \\
                        \hline
                    \end{tabular}
                    \centering
                    \caption{Přehled a srovnání modelů generujících kód}
                    \label{tab:code_models_comp}
                \end{table} 
            \end{landscape}

    % Reference
    \newpage
    \bibliographystyle{ieeetr}
    \bibliography{refs}
	
\end{document}
