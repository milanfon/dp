\newacronym{put}{PUT}{Parameterized Unit Test}
\newacronym{gui}{GUI}{Graphical User Interface}
\newacronym{gguf}{GGUF}{GPT-Generated Unified Format}
\newacronym{ggml}{GGML}{GPT-Generated Model Language}
\newacronym{ml}{ML}{Machine Learning - strojové učení}
\newacronym{vcs}{VCS}{Version Control System}
\newacronym{rnn}{RNN}{Recurrent Neural Network - Rekurenční neuronová síť}
\newacronym{cnn}{CNN}{Convolutional Neural Network - Konvoluční neuronová síť}

\newglossaryentry{llm}{
  name={LLM},
  description={Velký jazykový model (LLM - large language model) je typ modelu strojového učení, který je navržen tak, aby rozuměl a generoval text v přirozeném jazyce. Tyto modely jsou trénovány na obrovských datasetech a jsou schopny provádět různé úkoly, jako je textová klasifikace, generování textu, strojový překlad a další}
}

\newglossaryentry{testsmells}{
  name={testové pachy},
  description={Unit Test Smells jsou špatné návyky nebo postupy, které se mohou objevit při psaní jednotkových testů. Tyto "smelly" často vedou k neefektivním nebo nespolehlivým testům a mohou ztížit údržbu testovacího kódu. Příklady zahrnují Duplicated Asserts, Empty Tests a General Fixture},
}

\newglossaryentry{kontrakt}{
  name={kontrakt},
  plural={kontrakty},
  description={V kontextu jednotkových testů odkazuje na formálně definované podmínky nebo pravidla, které musí být dodrženy během vykonávání kódu. Kontrakty mohou specifikovat, co funkce očekává od svých vstupů a jaké výstupy nebo stavy by měla produkce kódu způsobit. Použití kontraktů pomáhá zajistit, že kód se chová podle očekávání a usnadňuje identifikaci chyb, když tyto podmínky nejsou splněny}
}

\newglossaryentry{filtr}{
  name={filtr},
  plural={filtry},
  description={Ve vývoji softwaru a testování se odkazuje na mechanismy nebo kritéria používaná k selektivnímu výběru testovacích vstupů nebo k rozhodování, které výstupy testů jsou relevantní pro další analýzu. Filtry mohou být použity k odstranění redundantních, nelegálních nebo jinak nežádoucích vstupů z procesu generování testů, což zvyšuje efektivitu testování tím, že se zaměřuje pouze na vstupy, které mohou odhalit chyby nebo porušení kontraktů}
}

\newglossaryentry{temperature}{
    name={teplota},
    plural={teploty},
    description={V terminologii strojového učení se jedná o hyperparametr, který ovládá reativitu a náhodnost vstupu modelu. Vyšší teplota vede k méně předvídatelným a kreativnějším výsledkům, kdežto nižší teplota produkuje více konzervativnější výstup.}
}

\newglossaryentry{context}{
    name={kontext},
    description={Rozsah textu nebo dat, který může model zpracovat při generování odpovědí. Funguje jako krátkodobá paměť modelu a určuje, kolik informací může model vzít v úvahu při tvorbě odpovědí.}
}

\newglossaryentry{token}{
    name={token},
    plural={tokeny},
    description={V NLP oblasti se jedná o rozdělení sekvence textu na menší části (například slova, části slov, atd.).}
}

\newglossaryentry{lsp}{
    name={LSP},
    description={Language Server Protocol. Protokol pro jazykový server používaný pro validaci syntaxe, sémantiky nebo formátování zdrojového kódu. Často se pod tímto označením myslí i samotný jazykový sever, který se o tuto analýzu stará.}
}

\newglossaryentry{servlet}{
    name={servlet},
    description={Serverový programový modul napsaný v programovacím jazyce Java, který zpracovává požadavky klientů a implementuje rozhraní Servlet. Přestože servlety mohou reagovat na jakýkoli typ požadavku, nejčastěji se používají k rozšíření aplikací hostovaných webovými servery.}
}

\newglossaryentry{benchmark}{
    name={benchmark},
    description={Standardizovaný test nebo pokus určení k meření a srovnání výkonu sytémů, komponent či aplikací.}
}
