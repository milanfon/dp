\newacronym{put}{PUT}{Parameterized Unit Test}
\newacronym{gui}{GUI}{Graphical User Interface}

\newglossaryentry{llm}{
  name={LLM},
  description={Velký jazykový model (LLM - large language model) je typ modelu strojového učení, který je navržen tak, aby rozuměl a generoval text v přirozeném jazyce. Tyto modely jsou trénovány na obrovských datasetech a jsou schopny provádět různé úkoly, jako je textová klasifikace, generování textu, strojový překlad a další}
}

\newglossaryentry{testsmells}{
  name={testové pachy},
  description={Unit Test Smells jsou špatné návyky nebo postupy, které se mohou objevit při psaní jednotkových testů. Tyto "smelly" často vedou k neefektivním nebo nespolehlivým testům a mohou ztížit údržbu testovacího kódu. Příklady zahrnují Duplicated Asserts, Empty Tests a General Fixture},
}

\newglossaryentry{kontrakt}{
  name={kontrakt},
  plural={kontrakty},
  description={V kontextu jednotkových testů odkazuje na formálně definované podmínky nebo pravidla, které musí být dodrženy během vykonávání kódu. Kontrakty mohou specifikovat, co funkce očekává od svých vstupů a jaké výstupy nebo stavy by měla produkce kódu způsobit. Použití kontraktů pomáhá zajistit, že kód se chová podle očekávání a usnadňuje identifikaci chyb, když tyto podmínky nejsou splněny}
}

\newglossaryentry{filtr}{
  name={filtr},
  plural={filtry},
  description={Ve vývoji softwaru a testování se odkazuje na mechanismy nebo kritéria používaná k selektivnímu výběru testovacích vstupů nebo k rozhodování, které výstupy testů jsou relevantní pro další analýzu. Filtry mohou být použity k odstranění redundantních, nelegálních nebo jinak nežádoucích vstupů z procesu generování testů, což zvyšuje efektivitu testování tím, že se zaměřuje pouze na vstupy, které mohou odhalit chyby nebo porušení kontraktů}
}
